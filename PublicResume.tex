%!TEX TS-program = xelatex
%!TEX encoding = UTF-8 Unicode

% Copyright 2017, Sean Bradly <sb@nsfw.jp>

\documentclass[]{SBResume}


% Personal Info for letterhead
%-------------------------------------------------------------------------------
%	PERSONAL INFORMATION - data common to various documents
%-------------------------------------------------------------------------------

\name{Sean}{Bradly}
\location{Austin, Texas}
\jobtitle{Security Consultant \:•\:\: Researcher \:\:•\:\: Developer }
\phonenumber{(512) 677-LULZ}
\emailaddr{sb@nsfw.jp}
\linkedin{sbradly}
\github{rhythmx}
\headshot{./images/profile2}




%-------------------------------------------------------------------------------
\begin{document}
\makeheader

\makeletterfooter{
  Complete work history and references are available upon request. For resume \textbf{\LaTeX}\ source, see \href{https://github.com/rhythmx/resume}{https://github.com/rhythmx/resume}
}

\begin{resume}
  
  \resumesection{About}
  \begin{resumetext}
    \begin{wrapfigure}{r}{3cm}
    \insertheadshot
    \end{wrapfigure}

    After 20 years of working in the realm of software engineering and 15 of
    those focused on entirely on cybersecurity, I have some level of experience
    in just about every facet of computing, and mastery in quite a few. My
    greatest skill, and why I currently find myself in consulting, is that I'm
    able to quickly absorb new subject matter and think outside the box.\\

    I've authored large software projects, audited source code and raw binaries
    for vulnerabilities, run penetration tests, reverse engineered firmware to
    verify features, rolled my own fuzzing frameworks, soldered this to that,
    wrote an operating system, and the list goes on. I have also honed my
    communication skills while authoring countless reports, whitepapers,
    presentations, and other media about all of the above, presenting them to
    coworkers, clients, and audiences at large for deep discussion.\\

    I was naturally drawn into the security space. I have always (ever since I
    could hold a screwdriver) liked to take things apart to see how they work. I
    like to imagine all the new and unexpected things that can be built out of
    all the bits and pieces. Working in computer security has always just that
    for me; breaking things down into components and getting to the ground truth
    of what is possible.\\

    \textbf{Skills:}
    \begin{resumeitemize2}
    \item{\textbf{General}: Security, R\&D, Reversing, Exploiting, Networking, Hardware, Automation}
    \item{\textbf{Languages}: C, C++, Ruby, Python, Assembly, Shell, Lisp, Java}
    \item{\textbf{Architectures}: X86, X86\_64, ARM, PIC, OpenRISC, Z80, Hexagon }
    \item{\textbf{Software}: Linux, Windows, GCC, Clang, GDB, Emacs, IDA, Ghidra, Burp}
    \end{resumeitemize2}
    
    \textbf{Personal Interests:}
    \begin{resumeitemize2}
    \item{\textbf{Aviation}: I'm currently working towards my private pilot's license}
    \item{\textbf{Cooking}: Modernist cuisine and science experiments for dinner!}
    \item{\textbf{Career}: Reading, tinkering, and attending conferences or local meetups to socialize and stay current}
    \item{\textbf{Other}: Music (guitars, bass, drums, et. al), woodworking, video games}
    \end{resumeitemize2}


  \end{resumetext}


\resumesection{Highlighted Projects}

\resumeentry
    {2014-2015}
    {
      \vspace{0.25cm}
      \begin{tikzpicture}%
        \node[circle, inner sep=0.6cm, fill overzoom image=images/google.png] () {};%
      \end{tikzpicture}
    }
    {Secure Embedded Operating System}
    {Inverse Limit (for Google)}
    {

        As part of Google ATAP's \emph{Project Vault}, Inverse Limit's
        small team of 4 designed and developed a complete computer
        platform with a security focus. All components are open
        source; the board schematics, OS, applications, drivers,
        toolchain, emulator, and even the CPU (based on OpenRISC) have
        been released to GitHub. Among other things, I was solely
        responsible for implementing the real-time multitasking
        operating system for the project.\\
%        
%        Earlier in the project I was responsible for creating the initial software that used a fake FAT filesystem as a means of socket communication between an external device and an Android device without need for a kernel driver.\\

        Launch video (Google I/O 2015): \hfill \href{http://goo.gl/5mZrVR}{\textbf{http://goo.gl/5mZrVR}}\\
        Source code: \hfill \href{http://goo.gl/0pbsk7}{\textbf{http://goo.gl/0pbsk7}}
    }
    
  \resumeentry
      {2013}
      {
        \vspace{0.20cm}
        \begin{tikzpicture}%
          \node[circle, inner sep=0.6cm, fill overzoom image=images/darpa.jpg] () {};%
        \end{tikzpicture}
      }
      {X86 Hypervisor and CPU Instruction fuzzer}
      {Inverse Limit (for DARPA)}
      {

        The MAIM project (Micro-architecture Instruction Mining) was
        one of three Cyber Fast Track proposals that DARPA accepted
        from Inverse Limit. It consisted of an x86 instruction fuzzer
        and a cross-platform hypervisor to execute the instructions
        and compare their behavior on different implementations. The
        project identified several undocumented differences between
        Intel, AMD and VIA architectures.\\

        Whitepaper: \hfill \href{http://goo.gl/Kwa3Rf}{\textbf{http://goo.gl/Kwa3Rf}}\\
      }
      
  \resumeentry
      {2010-2011}
      {
        \vspace{0.15cm}
        \begin{tikzpicture}%
          \node[inner sep=0.85cm,fill overzoom image=images/bpointsys.jpg] () {};%
        \end{tikzpicture}
      }
      {Complete TCP/IP Stack for Attack Traffic}
      {BreakingPoint Systems}
      {

        I redesigned BreakingPoint's existing network security test
        framework from a traffic simulator into a fully featured
        TCP/IP stack that could test live applications while
        transparently applying any number of advanced network evasion
        techniques. At the same time, by integrating concurrency into
        the new design and strategically replacing components with C
        extensions, the performance was quadrupled.}

      \newpage
      \:
\vspace{0.5cm}      
\resumesection{Work History}

  \resumeentry
      {2017-current}
      {
        \vspace{2.45cm}
        \fontsize{34pt}{1em}
        \begin{tikzpicture} % [x=3cm,y=4cm]%
          % \node[rectangle, inner sep=1.35cm,fill stretch image=images/atredis.png] () {};%
          \path[fill overzoom image=images/atredis.png] (0,0) rectangle (2.1cm,1.1cm);
        \end{tikzpicture}
      }
      {Atredis Partners, LLC}
      {Principal Research Consultant}
      {

        Atredis is an industry leader in security consulting. Focus is
        placed on bespoke consulting engagements, specifically tailored to the
        individual client's particular niche. As a Principal at Atredis, I am
        responsible for leading the technical effort as the main point of
        contact with clients and for organizing and directing the efforts of other
        members of the consulting team.\\

        % While Atredis' client list and any details of tests are strictly
        % confidential, here are some broad examples of projects I have completed
        % in my position:\\

        \resumeentryheading{Hardware / Embedded}
        \begin{resumemultiitem}
        \item{UEFI Reverse Engineering}
        \item{DSP Reverse Engineering (Hexagon)}
        \item{Secure Boot Reviews}
        \item{Hypervisor Testing and Source Reviews}
        \item{Baseboard Management Controller Reviews}
        \item{Device Driver Audits}
        \end{resumemultiitem}

        \resumeentryheading{Mobile}
        \begin{resumemultiitem}
        \item{Android Application Binary/Source Review}
        \item{Vendor HLOS Component/Configuration Review}
        \item{Trustlet Reverse Engineering}
        \item{Feature Development}
        \end{resumemultiitem}

        \resumeentryheading{Miscellaneous}
        \begin{resumemultiitem}
        \item{Technical Writing and Presentation}
        \item{Blockchain Application Reviews}
        \item{Custom Fuzzer Development}
        \item{Web Application Assessments}
        \item{Cloud Configuration Reviews}
        \item{Network Penetration Tests}
        \end{resumemultiitem}

      }
  \resumeentry
      {2013-2017}
      {
        \vspace{2.45cm}
        \fontsize{34pt}{1em}
        \:\:$\varprojlim$
      }
      {Inverse Limit, LLC}
      {Security Engineer and Researcher}
      {
        
      Inverse Limit was a research and engineering contracting company
      that was formed in 2013 ago with my colleagues Patrick Stach
      and Tim Carstens, supported by notable clients such as Google
      and DARPA.\\

      \resumeentryheading{Project Vault}
      \begin{resumemultiitem}
      \item{Embedded OS (see projects)}
      \item{Developed IO model using FAT filesystem}
      \item{Drivers for Custom Hardware}
      \item{Android Prototype Application}
      \item{Designed entire project layout and build system}
      \item{Hardware and Software completely open source}
      \end{resumemultiitem}

      \resumeentryheading{Project MAIM}
      \begin{resumemultiitem}
      \item{Custom x86 hypervisor (see projects)}
      \item{Co-Implemented x86/x64 instruction fuzzer}
      \item{Implemented main data analysis engine}
      \item{Authored final report with all research results}
      \end{resumemultiitem}

      \resumeentryheading{Other}
      \begin{resumemultiitem}
      \item{Design of new research proposals}
      \item{Created Kerberos protocol analysis tools}
      \item{Maintained libClang static code analysis tools}
      \end{resumemultiitem}
    }
  \resumeentry
      {2011-2013}
      {
        \vspace{1.20cm}
        \begin{tikzpicture}%
          \node[inner sep=1.35cm,fill overzoom image=images/levsec.png] () {};%
        \end{tikzpicture}
      }
    {Leviathan Security Group, Inc.}
    {Security Consultant and Developer}
    {

      Leviathan was (at the time) a small consultancy focusing on
      challenging niche projects. I was brought on to help facilitate long-term
      R\&D projects and to assist the consulting group as needed.\\
      
      \resumeentryheading{Mayor Myer (DARPA Research Program)}
      \begin{resumemultiitem}
      \item{Created polymorphic x64 shellcode encoder}
      \item{GNU libc heap corruption detector}
      \item{Intel JIT compiler and emulator}
      \end{resumemultiitem}
      \resumeentryheading{Consulting}
      \begin{resumemultiitem}
      \item{Audited Intel ME firmware}
      \item{Developed fuzzer for Intel ME applications}
      \item{Audits of embedded Java applications}
      \item{Android Research}
      \end{resumemultiitem}
      
    }
  \resumeentry
      {2007-2011}
      {
        \vspace{0.72cm}
        \begin{tikzpicture}%
          \node[inner sep=1.3cm,fill overzoom image=images/bpointsys.jpg] () {};%
        \end{tikzpicture}        
      }
    {BreakingPoint Systems Inc. (now Keysight)}
    {Security Engineer}
    {

      BreakingPoint's product is designed to be an ultra-high
      performance tool for testing network devices. It generates
      realistic network traffic at 100+ gigabits per second while
      monitoring the device under test for reporting.\\
      
      \begin{resumeitemize}
      \item{Complete TCP/IP implementation in Ruby (see projects)}
      \item{Implemented framework for network traffic simulation}
      \item{Discovered and reported new 3rd-party vulnerabilities}
      \item{Performed differential patch analysis on Microsoft updates monthly}
      \item{Maintained product coverage of important security vulnerabilities} 
      \item{Sample blog posts: \href{http://goo.gl/8yzJFv}{\textbf{http://goo.gl/8yzJFv}} - \href{http://goo.gl/GnWZGX}{\textbf{http://goo.gl/GnWZGX}}  }
      \end{resumeitemize}

    }
    \iffalse % skip complete work history by default
    \resumeentry
        {2006-2007}
        {\ }
    {Secured Infrastructure Design Corp. (now defunct)}
    {Security Engineer}
    {

      SIDC was a small security consulting firm based out of Tokyo,
      Japan that was developing a web portal allowing customers to
      setup daily automated security scans.\\
      
      \begin{resumeitemize}
        \item{Assisted development of automated vulnerability scanner}
        \item{Designed distributed TCP/UDP port scanner}
        \item{Implemented raw networking C extension for Python}
        \item{Relocated to Tokyo for 6 months} % (日本大好き!)
      \end{resumeitemize}
    }
  \resumeentry
    {2004-2006}
    {
      \vspace{0.72cm}
      \begin{tikzpicture}%
        \node[inner sep=1.05cm,fill overzoom image=images/gtech.png] () {};%
      \end{tikzpicture}        
    }
    {GTECH Corp.}
    {Automation Engineer / Systems Administrator}
    {

      About half (at the time) of the state lotteries in the US were
      managed by GTECH. As a Systems Administrator at their Austin
      datacenter, I was directly responsible for the ultra-high
      availability lottery systems of Texas, California, Idaho,
      Kansas, Jamaica, and Washington.\\
      
      \begin{resumeitemize}
      \item{Maintained and operated high volume lottery servers that handle thousands of transactions per minute in an environment where large government fines are imposed for downtime}
      \item{Administrated OpenVMS, AIX, Linux, Tru64  Unix, Windows (98, 2000, XP, 2003), MSSQL, and Sybase}
      \item{Automated a large part of the Operations department using Perl, Shell (csh and dcl), and BMC Control-M}
      \end{resumeitemize}
    }
    \resumeentry
        {2002-2004}
        {\ }
        {Manning Environmental Inc.}
        {Software Engineer}
        {
          
          Manning is a small, family-owned business in my hometown that
          designs and manufactures automatic fluid samplers; essentially
          robots that periodically pump water from a water treatment
          system into containers to send to a laboratory.\\
          
          \begin{resumeitemize}
          \item{Embedded programming for Zilog and Microchip architectures}
          \item{Maintained company web site, and several other sites on contract}
          \item{Circuit design verification and debugging}
          \item{Designed a low-cost microphone-based fluid detection circuit}
          \item{Designed production tests for new circuit boards}
          \end{resumeitemize}
        }
    \resumeentry
        {2000}
        {\ }
        {University of Texas: Applied Research Labs}
        {Java Programmer (paid intern)}
        {
          
          I obtained this summer programming internship while still in
          high school, learned Linux and Java while on-the-job, and
          completed all of my tasks successfully.\\
          
          \begin{resumeitemize}
          \item{Updated Java components for a U.S. Department of Defense project}
          \item{Assembled and installed new employee PC workstations}
          \item{Assisted staff with physical labor of office relocation}
          \end{resumeitemize}
        }
    \fi
\end{resume}
\end{document}
